\documentclass[12pt]{article}
\usepackage[osf]{mathpazo}
\usepackage{ms}
\usepackage[numbers]{natbib}
\usepackage{lineno}
\usepackage{graphicx}
\usepackage{caption}
\modulolinenumbers[5]
\linenumbers

\makeatletter
\renewcommand{\@biblabel}[1]{\quad#1.}
\makeatother

\title{How much of the world is woody?}
\author{
Richard G. FitzJohn$^{*,1,2}$, Matthew W. Pennell$^{*,3,4}$, Amy E. Zanne$^{5,6}$,\\ Peter F. Stevens$^{7}$, David C. Tank$^{3,8}$, William K. Cornwell$^{9}$
}
\date{}
\affiliation{\noindent
$^*$ These authors contributed equally\\
$^1$ Biodiversity Research Centre and Department of Zoology,
University of British Columbia, Vancouver, BC V6G 1Z4, Canada \\
$^2$ Department of Biological Sciences, Macquarie University, Sydney, NSW 2109, Australia \\
$^3$ Institute for Bioinformatics and Evolutionary Studies, University of Idaho, Moscow, ID 83844, U.S.A.\\
$^4$ National Evolutionary Synthesis Center, Durham, NC 27705, U.S.A.\\
$^5$ Department of Biological Sciences, George Washington University, Washington, D.C. 20052, U.S.A.\\
$^6$ Center for Conservation and Sustainable Development, Missouri Botanical Garden, St. Louis, MO, 63121, USA \\
$^7$Department of Biology, University of Missouri, St. Louis, MO 63166, U.S.A.\\
$^8$ Forest, Rangeland, and Fire Sciences Department and Stillinger Herbarium, College of Natural Resources, University of Idaho, Moscow, ID 83844, U.S.A.\\
$^9$ Department of Systems Ecology, VU University, 1081 HV Amsterdam, The Netherlands
}
\runninghead{How much of the world is woody?}

\begin{document}
\mstitlepage
\parindent=1.5em
\addtolength{\parskip}{.3em}

\begin{abstract}
  % RGF: Not sure about the first lines of this -- could/should be
  % more grabby.  Is the first sentence really a better lead than the
  % second?  Alternatively, can we start with the question (What
  % percentage of species are woody?).  Otherwise it's probably good.
  % 
  % WKC: needs to be 200 words
  % 
  Through large databases and synthetic research, a picture of global
  biodiversity is beginning to emerge, but especially with regard to
  global functional diversity, basic questions remain.  The
  woody/herbaceous divide is perhaps the most fundamental axis of
  functional diversity among plants.  We sought to address a very
  basic question: what percentage of the world's species are woody?
  Surprisingly, this question has never been comprehensively
  addressed.  With a database of woodiness for 49,064 species
  of vascular plants we estimated the status of the remaining
  % RGF: Number was incorrect: now should add up to 284,732
  % (284732 - 49064 == 235668)
  235,668 species using two Monte Carlo approaches.  We estimate the
  proportion ``woodiness'' among the world's vascular plants to be
  between 45\% and 48\%.  We also sought to determine whether a consensus
  answer to our question existed among botanists --- even if not
  formally documented in the peer-reviewed literature.  To investigate
  this, we coupled our analysis with a survey in which we posed our
  question to the broader community of botanists and other biologists.
  We found that no expert consensus existed and that, on average,
  researchers underestimated the proportion of the world's flora that
  was woody.  The results of our survey highlight that global datasets
  can show patterns at odds with conventional wisdom.
\end{abstract}

\newpage
\section{Introduction}

The distinction between a woody and non-woody growth-form of plants is
probably the most profound contrast among terrestrial plants and
ecosystems: the difference between a forest and a grassland is the
presence or absence of wood. The recognition of the fundamental
importance of this divide dates back at least to \textit{Enquiry into
 Plants} by Theophrastus of Eresus (371 -- 287 BC), a student of
Plato and Aristolte, who began his investigation into plant form and
function by classifying the hundreds of plants in his garden into
woody and herbaceous categories \citep{theophrastus1916enquiry}.

The last two thousand years of research into wood since Theophrastus
classified his garden have uncovered its origin in the early Devonian
($\sim$~400 Mya) \citep{gerrienne2011simple}; that prevalence of
woodiness varies with climate \citep{Molesheihgt}; that wood has been
lost many times in diverse groups, both extant and extinction
\citep{judd1994}; that many different ways of pseudo-woody growth
habit have appeared across groups that have lost true woodiness, or
diverged before true woodiness evolved \citep{Cornwellwood}.  We know
about its mechanical properties and developmental pathways, its
patterns of decomposition and their effects on ecosystem function
\citep{Cornwellwood}.  Woody and herbaceous species have different
rates of molecular evolution \citep{SmithDonoghue}.
% wkc: changed smith donoghue ref to be more precise
% RGF: Any reason for dropping the diversification?  It's in the same
% paper, from memory (sister clade contrast), and a pair of facts
% there helps the flow.
However, we have no idea about what proportion of species are
actually woody.

With the rise of bioinformatics, the similar biodiversity question of
``how many species are there on earth'' has been approached with a
variety of different methods \citep{may1988many,erwin1991many,
  stork1993many, joppa2010, costello2011, mora2011plos}.  These
statistical estimates are improved, narrowing in most recently on a
number around 8.7 million species with around 300,000 of those plants
\citep{mora2011plos}, although considerable variation still exists
among estimates \citep{costello2011}.

Here, we aim to use similar tools to re-ask Theophrastus' 2000 year
old question at the global scale: how many of the world's plant
species are woody?
%
We felt that such a basic question would be essentially known.
However, because we found little research that addressed this question
we took the unconventional approach of coupling our analysis of the
data with a survey in which we asked our question to the broader
community of botanists and other biologists.  The motivation for this
was to determine whether a consensus answer did exist --- even if not
formalized in the literature --- and if so, whether it was consistent
with the data.
% 
The tremendous variety of answers in response suggesting that this
fundamental aspect of plant functional diversity was surprisingly
poorly understood.

\section{Methods and Materials}

\subsection{Dataset}

% RGF: "largest database": of what, exactly?  Plant functional traits?
% If nobody is looking at estimating what fraction of species are
% woody, then it's not surprising that this is the largest such
% database.  If we just want to say that it's a really large database,
% we could say that this includes information on XX% of all extant
% species, perhaps? (this point still pending)

% WKC:yes, still need to finalize some decisions about this related to
% Amy/Jeremy I have re-written this again

We used a recently assembled database with growth-form data for
49,064 species, which is the largest database assembled to date.
% RGF: Alternatively: "49,064 species, representing almost 18\% of
% known the 274,141 vascular plant species."
%
% path.forest <- readLines("~/.forest_path")
% dat <- read.csv(file.path(path.forest, "export/speciesTraitData.csv"), stringsAsFactors=FALSE)
% sum(!is.na(dat$woodiness))
% 
% RGF: Will -- can you please fill something in here:
The core of the data is a growth form dataset from Kew Botanic Gardens
(REF?) with major additions from \citep{Molesheihgt} and \cite{apweb}.
This dataset takes a ``functional'' view of woodiness, which is
defined as the presence of a prominent, perennial, above-ground stem.
In addition to species producing true wood (i.e., consisting of
secondary xylem), this definition includes palms, tree ferns, and
bamboo as woody.  The complete implementation of the ``woody''
definition and its mapping to the Kew growth form categories is in
SUPPLEMENTAL TABLE CSV?.  The full dataset is %% will be
available via Dryad (reference to be added).  Taxonomically, we
included vascular plants (i.e., ferns, fern allies, gymnosperms, and
angiosperms).

Theophrastus both recognized the fundamental importance of the
distinction between woody and herbaceous plants, and that this
distinction is in some cases difficult to make.  There are a number of
species which are intermediate in form \citep{beaulieuHiddenRates}; in
this dataset 633 species (1.3\%) were coded as variable.  For the
formal analysis we excluded these species. Because the effort to
organize plant taxonomy, especially synonymy, is on-going, there was
uncertainty regarding the status of many of these plant names.  We
matched this database with the accepted names from the plant list
\citep{ThePlantList} leading to 37,488 records with documented
taxonomy---15,657 herbs and 21,831 woody species.  This included
records from all plant orders currently accepted by APG-III
\citep{APG3} along with the ferns taxonomy from ref \cite{apweb}.
% sum(dat.g$K) # known species
% sum(dat.g$H) # herbs
% sum(dat.g$W) # woody species

% RGF: Describe how we sorted out fern synonomy and got the accepted
% number of species?  Or was that just the allies?  Or just the higher
% order taxonomy.

% WKC: compare family sampling to total number of families? Somehow
% show the dataset is comprehensive?

\subsection{Estimating the percentage of species that are woody}

To estimate the percentage of species that are woody, we cannot simply
use the fraction of species within our trait database (21,831 / 37,488
= 58\%) as these records represent a biased sample of vascular plants.
% sum(dat.g$W) # woody
% sum(dat.g$K) # known
% sum(dat.g$W) / sum(dat.g$K) # fraction
For example, most Orchidaceae are probably herbaceous; we have only
two records of woodiness among the 1,574 species for which we have
data.
% i.orc <- dat.g$family == "Orchidaceae"
% sum(dat.g$K[i.orc]) # known Orchidaceae
% sum(dat.g$W[i.orc]) # woody Orchidaceae
However, the fraction of Orchidaceae with known data (1,574 / 27,104 =
6\%)
% sum(dat.g$K[i.orc]) # known Orchidaceae
% sum(dat.g$N[i.orc]) # number of Orchidaceae
% sum(dat.g$K[i.orc]) / sum(dat.g$N[i.orc]) # knowledge rate of Orchidaceae
is much lower than the overall rate of knowledge for all vascular
plants (36,238 / 274,141 = 13\%), which will downwardly bias the
global estimate of woodiness.
% sum(dat.g$K) # known vascular
% sum(dat.g$N) # total vascular
% sum(dat.g$K) / sum(dat.g$N) # knowlege rate for all vascular plants
%
Conversely, systematic under-sampling of tropical species, believed to
be more woody than temperate species \citep{Molesheihgt}, would bias
the global woodiness estimate downwards.

To impute the state (i.e., woody or herbaceous) of species with
unknown states, we used a simple Monte Carlo method to sample species
states.  We either assumed that (a) the observed species were drawn
with no replacement from a pool of species while making no assumption
about the distribution of states within the pool (``weak prior'') or
(b) that the observed species were drawn from a very large pool of
species with a proportion of woody species equal to the proportion of
woodiness among known species within the genus (``strong prior'').
%
For genera with no known states, we sampled a woodiness proportion
from the empirical distribution of woodiness proportion for other
genera within the same order.
%
We repeated the above sampling approach 1,000 times and report means
and confidence intervals over this distribution.
%
This procedure is described in detail in the appendix to this paper.

\subsection{Survey}

% RGF: As someone else mentioned -- this is a bit odd, as it's the
% first mention of the working group (this sentence is my fault).
% 
% In the introduction, we motivated this by saying that we thought
% there might be consensus already.  Another way of framing this could
% be to say that the reason why there is no published work on this
% question is that everyone knows the answer; given we didn't we
% wanted to find out if that was actually the case.
%
% Anyway, I suspect that everyone else will pick up on this, so I'd
% like to see what the other suggestions are before adding another
% version.
We were surprised to find little consensus within the members of our
NESCent working group, so proceeded to investigate if a general
consensus among biologists existed.
% 
In a similar study, botanical experts largely agreed with the
model--estimated number of species in several Angiosperm groups
\citep{joppa2010}, and we were interested in whether there was similar
anecdotal knowledge for the relative prevalence of this key trait.
%
We created an English-language survey asking for an estimate of the
percentage of species that are woody according to the above definition.  We also asked respondents to
indicate their level of familiarity with plants, level of formal
training, and the country in which they received their training (see
supplementary material \ref{fig:survey-text} for details).
%
We distributed the survey to the community via several electronic
mailing lists with wide circulation among biologists: \emph{EvolDir},
\emph{ECOLOG}, \emph{r-sig-phylo}, \emph{Taxacom}, \emph{Herbaria}, as
well as local lists. We also posted links on the social-networking
platforms \textsc{google+}, \textsc{twitter} and \textsc{facebook} to
reach a broad audience.
%
In order to increase representation of survey responses from Latin
America, we translated the survey into Portuguese (thanks to Rafael
Maia) and distributed it to Brazilian biology
\textsc{facebook} groups and university mailing lists.

To analyze the survey data, we used linear regression on
logit-transformed percent woodiness as (for discussion as to why we
used logit--transform, see ref \citep{wartonarcsine}) and treated the
self-reported level of botanical familiarity and education as factors.
All analyses were conducted with R \citep{R}.

\section{Results and Discussion}

Across all vascular plants, we estimated the fraction of woody species
to be between 45\% to 48\%.
% round(100*res.strong$overall, 1)
% round(100*res.weak$overall, 1)
Specifically, using our ``strong prior'' sampling approach we
estimated 45.0\% (95\% confidence interval of 44.7--45.3\%) and with
the ``weak prior'' approach we estimated 47.2\% (95\% CI of
46.6--47.8\%).
% 
The different approaches generate different distributions of the
per-genus percentage of woodiness (Figure
\ref{fig:distribution-genera}b), with a less strongly bimodal
distribution using the ``weak prior'' approach, as we describe in the
appendix.
%
This difference in the predicted distribution drives the difference in
the estimates of woodiness.  Genera that are primarily herbaceous
(percentage of woody species less than 10\% for genera with at least
10 records) were on average larger than primarily woody genera (more
than 90\% woody species), with 202 species rather than 138.

Higher--order classifications are at least as much a product of human
pattern matching as any biological process.  Genera probably
correspond to the morphological discontinuities that humans deem
important \citep{scotland2004significance}, which likely includes
woodiness.  The relative paucity of genera with significant numbers of
both woody and herbaceous species (Figure
\ref{fig:distribution-genera}) reinforces the importance of this
trait.

Because the observed distribution of woodiness fraction among genera
is itself very bimodal, we are inclined to believe that the true
result lies closer to 45\% than to 47\%.  A more sophisticated
hierarchical modeling approach could lead to a more precise
answer, but we feel that our values probably span the range of
estimates that such an approach would generate.
%
Probably a larger problem is that undiscovered species are unlikely to
be random with respect to woodiness.  In principle, rarefaction
analysis could estimate the number of species remaining to be
discovered in different groups, but this is not possible for many
plant clades \citep{costello2011}.
%
In any case, we feel that such a simple question as this deserves a
simple answer.

The distribution of woodiness varies widely among different taxonomic
groups (Figure \ref{fig:phylogeny} and ref
\citep{sinnott1915evolution}).  There are two major clades that are
primarily herbaceous --- the monocots and ferns
(Monilophyta). However, there are many primarily herbaceous clades
nested within woody clades, and vice versa, which makes the
combination of taxonomic and functional information crucial for
answering this type of question.  The two different approaches (strong
versus weak priors) led to similar taxonomic distributions of
estimated woodiness (Figure \ref{fig:phylogeny} versus Figure
\ref{fig:phylogeny-supp}), differing only in detail.

In contrast with the very small range in our estimate of the
percentage of woody species in the world, there is strikingly little
consensus among researchers.  We received 293 responses from 31
countries, which ranged from estimates of 1\% to 90\% with a mean of
31.7\%.  The vast majority (81\%) of the responses were less than even
our smallest estimate.
% nrow(d.survey)
% range(d.survey$Estimate)
% mean(d.survey$Estimate < 45)
We found little effect of a respondent's level of training on their
estimate (Figure \ref{fig:survey}).  There was a significant effect of
the respondent's familiarity with plants on the estimates, primarily
driven by respondants with little botanical familiarity (the ``What's
a Plant?'' category), whose estimates tended to be lower (less
woody) than the estimates from other categories.
% 
Before carrying out the survey, we had hypothesized that researchers
from tropical regions may perceive the world as woodier than
researchers from more temperate regions due to the latitudinal
gradient in woodiness \citep{Molesheihgt}.  However, in our informal
survey, respondents who trained in tropical countries were no more
accurate in their predictions than their temperate colleagues (results
not shown).
% RGF: Is the code for doing the classification handy anywhere?  The
% country list is rather long.  Even though the results aren't shown,
% I'd like to add them to the analysis file for the record.

We conclude with two observations.
%
First, most people perceive there to be substantially fewer woody
species in the world than there probably are.  This herb-centric view
of the world may arise from the importance of our (mostly herbaceous)
cultivated crops, or the fact that people --- including researchers
--- likely spend more time in the garden than in the forest.
%
Second, there is no consensus as to how woody the world is.  We find
it very surprising that this most basic aspect of plant natural
history was not generally known.  Global plant functional diversity is
much woodier than we thought.

\section{Acknowledgments}

We thank the members of the Tempo and Mode of Plant Trait
Evolution working group for contributing to project development,
members of the broader community who took the time to fill out and
comment on our survey and Rafael Maia for translating our survey and
helping us to distribute it.
% RGF: Hiding this down here a bit:
We thank Dales Indian Cuisine in Durham, NC for providing the buffet
lunch over which this project was brought to life.
%
This work was supported by the National Evolutionary Synthesis Center
(NESCent), NSF \#EF- 0905606, Macquarie University Genes to Geoscience
Research Centre through the working group.
%
RGF was supported by a Vanier Commonwealth Graduate Scholarship from
the Natural Sciences and Engineering Research Council of Canada
(NSERC).
% RGF: Don't you have NSERC as well, Matt?
MWP was supported by a NESCent graduate fellowship.
%
WKC was supported by Netherlands Organisation for
Scientific Research (NWO) through its Open Competition Program of the
section Earth and Life Sciences (ALW) grant nr. 820.01.016.

\bibliographystyle{prsb}
\bibliography{wood.bib}

\begin{figure}[p]
  \centering
  \includegraphics{figs/fraction-by-genus}
  \caption{Distribution of the pertange of woodiness among genera.
    The distribution of the percentage of species that are woody within
    a genus is strongly bimodal among genera (panel a -- showing
    genera with at least 10 species only).
    % 
    The two different sampling approaches generate distributions that
    differ in their bimodality (panel b). When assuming species are
    sampled with replacement from some pool, but having no prior on
    the fraction of woodiness within the pool generates a broad
    distribution with many polymorphic genera (blue lines), while
    sampling with replacement assuming that species are drawn from a
    pool of species that has a fraction of woody species equal to the
    observed fraction of woodiness generates a strongly bimodal
    distribution (red lines).}
  \label{fig:distribution-genera}
\end{figure}

\begin{figure}[p]
  \centering
  \includegraphics{figs/fraction-on-phylogeny}
  \caption{Distribution of the percentage of woodiness among orders of
    vascular plants.  Each tip represents an order, with the width of
    the sector proportional to the log number of recognised species in
    that order (data from accepted names in ref \citep{ThePlantList}).
    The bars around the perimeter indicate the percentage of woody
    (blue) and herbaceous (orange) species, estimated using the
    ``strong prior'' (sampling without replacement) approach.  Using
    the ``weak prior'' approach generally leads to an estimated
    percentage that is closer to 50\% (see Figures
    \ref{fig:phylogeny-supp} and \ref{fig:distribution-genera}).}
\label{fig:phylogeny}
\end{figure}

\begin{figure}[p]
  \centering
  \includegraphics{figs/survey-results}
  \caption{Distribution of responses to the survey question ``What
    percentage of the world's vascular plant species are
    woody?''. Responses are broken up by familiarity with plants
    (panel (a)) and formal training in botany or a related discipline
    (panel (b)). The mean and 95\% confidence intervals for our
    estimates of the proportion of woody species from the empirical
    data are depicted by the horizontal shaded rectangles; the blue
    upper rectangle corresponds to the ``weak prior'' approach and the
    red lower rectangle corresponds to the ``strong prior'' approach
    (see Appendix for details).}
  \label{fig:survey}
\end{figure}

\clearpage
\renewcommand\thefigure{S.\arabic{figure}}
\appendix
\section{Appendix: Sampling proceedure}
\setcounter{figure}{0}    

Here, we describe our algorithm for sampling the states of species
that lack information on woodiness. We treat each genus separately,
and in all cases know that there are are $n_w$ woody and $n_h$ species
and a total of $N$ species in the genus.
%
For example, the genus \textit{Microcoelia} (Orchidaceae) has 30
species in total, and we know that 13 are herbaceous and none are
known to be woody ($N = 30$, $n_w = 0$, $n_h = 13$). We do not know
the state of the remaining 17 species, so the true number of woody
species, $N_w$, must lie between 0 and 17. In general, we cannot
assume that these species are all herbaceous, even though both
biological and mathematical intuition suggest that most of them will
be.

We used two different approaches for imputing the values of these
unknown species. First, we assumed that the known species were
sampled without replacement from a pool of species with $N_w$ woody
and $N_h$ herbaceous species ($N_w + N_h = N$). The probability that
$x$ of the species are woody ($x = 0, 1, \ldots, N - n_w - n_h$) is
proportional to
\begin{equation}
  \Pr(N_w = x) \propto {n_w + x \choose n_w}
  {N - n_w - x \choose n_h}
\end{equation}
For \textit{Microcoelia} this gives a 45\% probability that all
species are herbacious, and a 92\% chance that at most three species
are woody.

This approach probably overestimates the number of woody species in
this case, and in other cases where all known species are woody (e.g.,
\textit{Actinidia} [Ericaceae]) it will probably underestimate the
number of species that are woody. We see this as corresponding to a
very weak prior on the shape of the distribution of the fraction of
woody species within a genus.

However, the distribution of woodiness among genera and families is
strongly bimodal. Most genera are either all-woody or all-herbacious
(Figure \ref{fig:distribution-genera} and ref
\citep{sinnott1915evolution}).  Among the 748 genera with at least 10
records, 392 are entirely woody, 248 are entirely herbaceous, and only
50 have between 10\% and 90\% woody species. As a result, knowing the
state of a handful of species within a genus can give a reasonable
guess at the remaining species.
% tmp <- dat.g$p[dat.g$K >= 10] # genera with 10 records
% sum(tmp == 1) # 100% woody
% sum(tmp == 0) # 100% herbaceous
To model the other extreme of sampling, we used an approach where we
computed the observed fraction of woody species ($p_w = n_w / (n_w +
n_h)$) and sampled the state of the unobserved species using a
binomial distribution, so that the probability that $x$ of the species
are woody is
% RGF: I need to check this when not sleep-deprived/jet lagged. This
% should be an offset binomial distribution. It may be clearer if we
% define things about numbers of known and unknown species.
\begin{equation}
  \Pr(x = k) = {N - n_w - n_h \choose k - n_w - n_h} 
  p^k (1-p)^{N - n_w - n_h - k}.
\end{equation}

In cases where all known species are woody (or herbaceous) this will
assign all unknown species to be woody (or herbaceous). In polymorphic
cases this will give similar results to the hypergeometric sampling
approach above. This approach corresponds to a very strong prior on
the shape of the distribution of woodiness among genera.
% RGF: This is really vague and I will think about it more when I'm
% less braindead.
While neither of these approaches is ``correct'', they probably
span the range of possible outcomes.

For genera where there was no information on woodiness for any
species, we sampled a fraction of species that might be woody from the
empirical distribution of woodiness fractions \textit{among genera}
within the same order. We did this after imputing the missing species
values within those other genera. So, if a genus is found in an order
with genera that had woodiness fractions of $\{0, 0, 0.1, 1\}$ we would
have approximately a 50\% chance of sampling a 0\% woodiness fraction
for a genus, with probabilities from 0.1 to 1 being fairly evenly
spread.  Given this woodiness fraction, we then sampled the number of
species that are woody from a binomial distribution with this fraction
and the number of species in the genus as its parameters.

\begin{figure}[p]
  \centering
  \includegraphics{figs/fraction-on-phylogeny-supp}

  \caption{(Supplementary)
    \textit{This is Figure \ref{fig:phylogeny} using the alternative
      sampling approach.}\\
    %
    Distribution of the fraction of woodiness among orders of vascular
    plants.  Each tip represents an order, with the fraction of
    circumference proportional to the log number of recognised species
    in that order (data from accepted names in ref
    \citep{ThePlantList}).  The bars around the perimeter indicate the
    percentage of woody (blue) and herbaceous (orange) species,
    estimated using the ``weak prior'' (sampling with replacement)
    approach.  Using the ``strong prior'' approach generally leads to
    an estimated percentage that is further away from 50\% (see
    Figures \ref{fig:phylogeny} and \ref{fig:distribution-genera}).}
  \label{fig:phylogeny-supp}
\end{figure}

\begin{figure}[p]
  \centering
  \includegraphics{figs/distribution-raw}  
  \caption{(Supplementary) The posterior probability distribution for
    the proportion of the world's flora that is woody, using our two
    sampling approaches.  The red (left) distribution samples missing species
    with replacement (strong prior), while the blue distribution
    (right) samples missing species without replacement (weak prior).
    See Appendix for details.}
  \label{fig:distribution-raw}
\end{figure}

\begin{figure}[p]
  \centering
  \vspace{-20ex}
  \includegraphics[scale=0.7]{figs/Survey_supplemental}
  \caption{(Supplementary) English-language version of the survey we
    distributed}
  \label{fig:survey-text}
\end{figure}

\begin{figure}[p]
  \centering
  \includegraphics{figs/survey-distribution}
  \caption{(Supplementary) Distribution of all responses to the survey
    question ``What percentage of the world's vascular plant species
    are woody?''.
    %
    The mean and 95\% confidence intervals for our estimates of the
    proportion of woody species from the empirical data are depicted
    by the horizontal shaded rectangles; the blue rectangle corresponds
    to the ``weak prior'' approach and the red rectangle corresponds
    to the ``strong prior'' approach (see Appendix for details).  }

  \label{fig:survey-distribution}
\end{figure}

\end{document}

%%% Local Variables:
%%% mode: latex
%%% TeX-master: t
%%% TeX-PDF-mode: t
%%% End:
