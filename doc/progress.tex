\documentclass[12pt]{article}

\usepackage{mathpazo} % less ugly fonts
\usepackage{minionpro} % really nice fonts

%\newcommand{\o}{\ensuremath{_{o}}}
\newcommand{\fo}{\ensuremath{_{fo}}}
\newcommand{\gfo}{\ensuremath{_{gfo}}}
\newcommand{\xgfo}[1]{\ensuremath{_{#1,gfo}}}
\renewcommand{\ss}{\sigma^2}

\newcommand{\mycomment}[1]{\begin{quote}[\textit{#1}]\end{quote}}

\begin{document}

\mycomment{This started as a sketch for the methods section, but now
  it illustrates the problem I have at the moment.}

\section{Simulation methods}

Our aim is to estimate the fraction of species in the world that are
woody.  We cannot simply use the fraction of species within our trait
data set (29,232 / 45,118 = 62.6\%) as these records may represent a
biased sample of vascular plants.  
%
For example, most Orchidaceae are probably herbaceous; we have no
records of woodiness for the 1,646 species recorded.  However, the
fraction of Orchidaceae with known data (1,646/27,115 = 6\%) is lower
than the overall average (45,118 / 276,373 = 16\%), so we need to
account for the possible states of missing species.
%
Conversely, systematic undersampling of tropical species, believed to
be more woody than temperate species (Jeremy's paper?), will bias the
woodiness estimate downwards.

Most genera are either all-woody or all-hebacious.  Among the 847
genera with at least 10 records, 480 are entirely woody, 254 are
entirely herbaceous, and only 66 have between 10\% and 90\% woody
species.  A slightly different pattern holds further up the taxonomic
heirarchy: some orders contain genera that are more bimodal than
others.

\mycomment{This is what we did at the last working group to get the
  47.5\% estimate}

We imagined a sampling process like this: suppose that for some genus
there is a true number of woody species $N_w$ among the $N$ species in
the genus, and that we have randomly sampled $n_w$ woody and $n_h$
species from this pool.  
%
For example, with the genus \textit{Microcoelia} (Orchidaceae) all 13
species with known state are herbaceous ($n_w = 0$, $n_h = 13$, but we
we do not know the state of the remaining 17 species in the genus ($N
= 30$).  In general, we can't assume that these species are all
herbacious, even though odds are that most of them will be, so the
true number of woody species may lie between 0 and 17.
%
This pattern of ``sampling without replacement'' can be modelled with
a hypergeometric distribution, so that the probability that $x$ of the
species are woody ($x = 0, 1, \ldots, N - n_w - n_h$) is proportional
to
\begin{equation}
  \Pr(N_w = x) \propto {n_w + x \choose n_w}
    {N - (n_w + x) \choose n_h}
\end{equation}

For \textit{Microcoelia} this gives a 45\% probability that all
species are herbacious, and a 92\% chance that at most three species
are woody.

\mycomment{And this is the problem\ldots}

However, this seems unlikely; of 450 genera in Orchidaceae,
\textit{all} have a point estimate of 0\% woody.  Therefore it seems
far more likely that the unknown \textit{Microcoelia} estimate should
all be herbacious.  This means that something about the distribution
of genus estimates within family should be informing the estimation of
states for missing taxa.

\mycomment{This is what we did for the remaining genera, but I still have
  not expanded this to try and give an intuitive explanation}

For genera with no known species, we computed the distribution of the
fraction of woodiness for other genera within the family and sampled a
woodiness fraction from this (empirical) distribution.

\mycomment{How I started proceeding}

Assume that the distribution of genus-level woodiness proportion
\textit{within a family} is representable by a Beta distribution with
parameters $\alpha$ and $\beta$:
\begin{equation}
  \Pr(p) \propto p^{\alpha - 1}(1 - p)^{\beta - 1}
\end{equation}

Then, given an estimate of the proportion of woody taxa $p$ for a
genus, the probability of our observed data within a genus is given by
the binomial distribution
\begin{equation}
  \Pr(n_w, n_h | p) \propto p^{n_w}(1 - p)^{n_h}
\end{equation}

I tried treating $\Pr(p)$ as a prior and $\Pr(n_w, n_h | p)$ as a
likelihood and constructing a MCMC chain to estimate the proportion of
woodiness for each genus.  This worked really well for the
Orchidaceae, giving estimates that were much closer to 0\% (I think
around an estimate of 50 undiscovered woody species over the whole
family).  The speed is helped by the fact that you can directly
sample the vector of $p$ values from the conditional distribution,
which turns out to be a Beta distribution itself.

However, for smaller families, such as the Winteraceae, the chain gets
stuck.  Basically the beta distribution $\Pr(p)$ moves to a space that
puts extremely high probability density on 0\% herbacious (other
groups will obviously go the other extreme), we end up sampling p
values that are indistinguishable from 1, and the likeihood becomes
infinite.  I tried having the parameters of $\Pr(p)$ be informed by
another Beta distribution at the order level, but that doesn't seem to
help.

\end{document}
