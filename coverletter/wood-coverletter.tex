\documentclass[a4paper,12pt]{article}

\setlength{\parskip}{1ex plus1pt} % threequarters
\RequirePackage[hmargin=3.5cm,vmargin=3cm]{geometry}
\usepackage{graphicx}
\pagestyle{empty}
\usepackage[osf]{mathpazo}

\begin{document}

{\raggedleft
  Dr. Richard FitzJohn\\
  Department of Biological Sciences\\
  Macquarie University\\
  Sydney, NSW 2109, Australia\\[2ex]
}

\vspace{3ex}

Dear editor,

Please consider our manuscript ``How much of the world is woody?''
for publication.  This might be seen as an unconventional paper, but
on the other hand this is a very basic question in global functional
plant ecology, and unanswered despite 2,000 years of research.

To solve the problem we used a recently assembled database and a
stochastic gap--filling approach.  This allows us to estimate both the
global proportion of species that are woody, along with uncertainty in
the estimate.  Our answer is 47 $\pm$ 2\%, which is much higher than
the conventional wisdom.

In many ways we feel this paper is the functional analogue to the series
of papers making statistical estimates on the number of species in the
world.  Like those papers, we feel that the simplicity of the question
posed in this paper may lead to wide interest in the result.

The dataset was assembled by Zanne et al. (in prep).  For the review
process we have made the genus-level data set and analysis script
available to reviewers so that they can fully replicate the analysis.
The full species-level dataset will be deposited at Dryad to be made
available on publication of the Zanne et al. paper.

We hope that you will consider our work for publication.

\vspace{2ex}
\hspace{.2\textwidth}Yours sincerely,\\[2ex]
\hspace*{.2\textwidth}
\includegraphics[height=9ex]{rich-sig}\\[2ex]
\hspace*{.3\textwidth}
Richard FitzJohn

\end{document}
